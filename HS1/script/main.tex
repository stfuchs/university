\documentclass[journal]{vgtc}
\let\ifpdf\relax
\usepackage{hs-vis_ss10}


%% Please note that the use of figures other than the optional teaser
%% is not permitted on the first page of the journal version.  Figures
%% should begin on the second page and be in CMYK or Grey scale
%% format, otherwise, colour shifting may occur during the printing
%% process.  Papers submitted with figures other than the optional
%% teaser on the first page will be refused.

%% These three lines bring in essential packages: ``mathptmx'' for
%% Type 1 typefaces, ``graphicx'' for inclusion of EPS figures. and
%% ``times'' for proper handling of the times font family.

\usepackage{mathptmx}
\usepackage{graphicx}
\usepackage{times}


%% allow for this line if you want the electronic option to work
%% properly
\vgtcinsertpkg


%% author name
\author{Steffen Fuchs}

%% paper title
\title{Objektsegmentierung im Video: Ein hierarchischer, variationaler Ansatz, um Punkttrajektorien auf dichte Regionen zu erweitern}

%% short title for header
\shorttitle{Visualisierung von Regenbogenfarben}


%% Abstract section.
\abstract{%
  Punkttrajektorien zeichneten sich in der Vergangenheit als leistungsstarkes Hilfsmittel ab, um unbeaufsichtigt qualitativ hochwertige
  Segmentierungen auf Videosequenzen durchzuführen. Sie können zwar die Langzeit-Bewegungsunterschiede zu ihrem Vorteil nutzen,
  liefern auf Grund ihres Rechenaufwandes und der Schwerigkeiten in homogenen Regionen in der Regel aber nur spärliche Informationsdichte.
  Die Authoren P. Ochs und T.Brox stellen mit ihrer Arbeit eine variationale Methode vor, um aus den Gruppierungen dieser grob verteilten
  Trajektorien eine dichte Segmentierung zu gewinnen.  Die Information wird dabei mittels eines hierarchischen, nicht-linearen Diffusionsprozesses
  propagiert, welcher zwar im kontienuierlichen Bereich arbeitet, jedoch Superpixel mit berücksichtigt.
  Es wird gezeigt, dass dieser Prozess Informationsdichte von 3\% auf 100\% erhöht, als auch die durchschnittliche Genauigkeit der Labels.
} % end of abstract


%% Uncomment below to include a (optional) teaser figure.
%\teaser{ \centering
%  \includegraphics[width=16cm]{images/CypressView.jpg}
%  \caption{In den Wolken: Vancouver von Cypress Mountain. Auf der
%    ersten Seite d"urfen keine Grafiken au"ser dieser optionalen
%    Aufmachgrafik (Teaser) abgebildet sein.}
%}


%%%%%%%%%%%%%%%%%%%%%%%%%%%%%%%%%%%%%%%%%%%%%%%%%%%%%%%%%%%%%%%%
%%%%%%%%%%%%%%%%%%%%%% START OF THE PAPER %%%%%%%%%%%%%%%%%%%%%%
%%%%%%%%%%%%%%%%%%%%%%%%%%%%%%%%%%%%%%%%%%%%%%%%%%%%%%%%%%%%%%%%%

\begin{document}

%% The ``\maketitle'' command must be the first command after the
%% ``\begin{document}'' command. It prepares and prints the title
%%   block.

%%   the only exception to this rule is the \firstsection command
\firstsection{Einleitung}
\maketitle
Aktuelle Lernansätze für visuelle Erkennung hängen sehr von der manuellen Markierung und Segmentierung von Objeckten ab.
Betrachtet man das bis heute beste visuelle Erkennungssystem - das menschliche Gehirn - wird klar,
dass solche manuellen Hilfestellungen nicht notwendig sind. Säuglinge erlernen die visuellen Formen und
Eigenschaften von Objekten auch ohne dass ihnen ihre Eltern Bounding Boxes darum legen oder eine Segmentierung zur Verfügung stellen.
Es gibt überzeugende Beweise darüber, dass Säuglinge dies Art von Objektsegmentierung durch Bewegungseinsatzen dürchführen \cite{}.
und man könnte letztendlich argumentieren, dass das rechnergestützte visuelle Systeme sich immer weiter dem menschlichen Sehvermögen annähren sollten.
\\
Die Bewegungsanalyse von Punkttrajektorien ist ein angemessens und robustes Werkzeug, um in Videosequenzen die Regionen von Objekten ohne
menschliches Zutun automatisch bestimmen und extrahieren zu können, wie in \cite{} erst kürzlich gezeigt.
Diese Ansätze verlangen jedoch danach, dass für die Bewegungsschätzung immer auch genügend Strukturen in den Bildern vorhanden sind,
zu denen Übereinstimmungen gefunden werden können.
In homogenen Gebieten gibt es diese Strukturen aber nicht, was dazu führt, dass die resultierenden Punkttrajektorien nur spärlich vorhanden sind.
In der Arbeit von \cite{} werden die Punkttrajektorien zwar aus dem dichten, optischen Flussfeld berechnet und resultierenden Trajektorien würden
ebenfalls für das gesamte Bild zur Verfügung stehen, jedoch sind diese in homogene Regionen weniger zuverlässig und können das Clustering behindern.
Zudem verlangen die eingeschränkte Verfügbarkeit von Rechenkraft nach der Reduzierung der Trajektorien, die analysiert werden sollen.
Das Clustern von dichten Punkttrajektorien würde viel zu lange dauern.
\\
Die Authoren stellten in ihrem Artikel eine Methode basierend auf der Variationsrechnung vor,
die aus wenigen Clustern von Punkttrajektorien eine dichte Segmentierung erstellt\ref{fig:01}.
Auf dem ersten Blick mag dies nach einem simplen Interpolationsproblem aussehen, da unser Verstand ganz einfach die Lücken zwischen den Punkten füllen kann.
Bei genaurer Betrachtung werden jedoch einige Schwierigkeiten deutlich.
So werden zum Beispiel einige der kritischen Bereich überhaupt nicht von den Trajektorien abgedeckt,
ganz besonders sind davon die Grenzen der Objekte betroffen. Den Trajektorien, die an den Grenzen jedoch vorhanden sind,
wurden in den meisten Fällen falsche Labels zugewiesen, da der zugrunde liegende optische Fluss gerade bei Okklusion ungenau wird.
Schlussendlich ist gerade in homogenen nahezu keine Information über die entsprechenden Labels vorhanden.
\\
Der Schlüssel, um die bestehenden Daten der Labels zuverlässing propagieren zu können, liegt darin, sich die Farb- und Kanteninformationen
zu nutze zu machen, was sich sehr gut mit der Trajektorienbestimmung ergänzt. Bemerkenswert ist, das Segmentierung basierend auf Farbdaten
am besten in homogenen Bereichen arbeitet, eben dort wo die Probleme der Bewegungsbasierten Segmentierung liegen.
Dies wird erreicht, indem die Informationen in Abhängigkeit der Farbhomogenität verbreitet wird. Am Ende wird ein hierarchischer variationaler Ansatz
vorgestellt, mit kontinuierliche Labelfunktion auf mehreren Ebenen. Jede Ebene entspricht einer Unterteilung in Superpixeln mit einem bestimmten
Grobkrörnigkeit. Im Vergleich zu einem Model mit nur einer Ebene stehen  Hilfsfunktionen auf den gröberen Ebenen zur Verfügung, die mittels eines
verbindenden Diffusionsprozesses optimiert werden.

Der Vorteil dieses Verfahrens liegt darin, dass das Propagieren der Labels, dank des hierarchischen Ansatzes, die Struktur berücksichtig, während Metrisierungsfehler und Blockartefakte vermieden werden können. Dieses treten besonders bei diskreten Markov Random Fields (MRF) Modellen auf.

\section{Verwandte Arbeiten}


Die hier behandelte Problematik ist sehr verwandt mit der interaktiven Segmentierung,
bei der der Benutzer selbst zunächst einfache Markierungen im Bild vornimmt
und anschließend propagiert das entsprechend gewählte Verfahren diese Labels zu den restlichen Stellen.
Mehrere existierende Techniken basiern auf Graph Cuts \cite{004} oder Random Walks \cite{012}.
Die aktuellsten Techniken bauen auf konvex relaxierte Variationsrechnungen \cite{023,018,014,016} auf, die es, im Gegensatz zu den klassischen,
auf einem Graph definierten MRF, vermeiden, Diskretisierungsartefakte zu erzeugen.
Die hier vorgestellte variationale Methode baut auf die Regularizierung von \cite{014} auf.

Keines der genannten Verfahren zieht einen hierarchischen Ansatz in Betracht. Vielmehr unterscheiden sich die Labels
der Punkttrajektorien von denen, die durch die Benutzer eingezeichnet werden, in zwei Dingen:
Zum einen sind erstere unbeaufsichtig generiert und damit mit großer Wahrscheinlichkeit fehlerbehaftet, während bei letzteren stets
von der Richtigkeit der Benutzereingabe ausgegangen wird. Dies bedeutet, dass man nicht von einem Interpolationsproblem ausgeht,
sondern ein Approximationsproblem vorliegt.
Zum anderen erhält man durch den Benutzer eine dichte und endliche Vorauswahl, während die Labels der Trajektorien sich nur aus einzelnen Punkten zusammen
setzen und über das gesamte Bild verteilt sind.

Das hier vorgestellte Model ist ebenfalls verwandt mit dem Bildkompressionsverfahren durch anistrope Diffusion \cite{011}. Dieses
Verfahren erhält nur eine kleine Menge der Pixelwerte der ursprünglichen Bildpunkte und versucht die übrigen mittels
Diffusionsprozess wieder herzustellen.

Desweiteren gibt es zahlreiche, aktuelle Arbeiten zur dichten Bewegungssegmentierung, die eine Übersegmentierung durch den Einsatz von
Superpixeln, Labelpropagierung mittels optischen Fluss oder anderen Clusterverfahren erzielen \cite{005,013,024,015}.
Diese liefern jedoch nicht die tatsächlichen Objektregionen.
Einige interaktive Videosegmentierungen versuchen dies zu vermeiden \cite{003,019}, jedoch verlangen diese Verfahren wiederum die Eingabe eines
Benutzers und laufen somit nicht unbeaufsichtigt ab.
\section{Einstufiges, variationales Model}

Wie in den vorangegangen Sektionen beschreiben stehen nun die erwähnten Punkttrajektorien und Labelinformation zur Verfügung. Diese seien
beschrieben als Labelfunktion $\tilde u := (\tilde u_1,\dotsc,\tilde u_n): \Omega \rightarrow  \{ 0,1 \}^n, n \in \mathbb{N}$, die $n$ verschiedene
Labels repräsentiert, wobei
\begin{equation}
  \tilde u_i := \left\{
    \begin{split}
      1, & \quad \mathrm{if}\; x\in L_i \\
      0, & \quad \mathrm{else }
    \end{split} \right.
\end{equation}
und $L_i$ das Set von Koordinaten ist, die von einer Trajektorien des Labels $i$ eingenommen werden, und $\Omega \subset \mathbb{R}^2$ den
Bildbereich beschreibt. Anschaulich gesprochen würde $\tilde u$ eine Menge von Binärbildern beschreiben, bei dem jedes $\tilde u_i$ für ein Label steht.
Der Einfachheit halber wird sich im Folgenden auf die Arbeit mit Einzelbildern beschränkt. Jedoch versichern die
Authoren, dass die Methoden auch ohne weiteres auf die Berechnung der ganzen Videosequenz erweitert werden kann.

Gesucht ist nun eine Funktion $u := (u_1,\dotsc,u_n): \Omega \rightarrow  \{ 0,1 \}^n$, die nahe an den Labels bleibt, die bereits in den Punkten in
$L := \cup_{i=1}^n L_i$ verfügbar sind. Dies wird erreicht, indem man versucht die Energie
\begin{equation}
  E_{\mathrm{data}}(u) := \frac{1}{2} \int_\Omega c \sum \limits_{i=1}^n (u_i - \tilde u_i)^2 dx
\label{eq:data1}
\end{equation}
zu minimieren. Dabei ist $c:\Omega \rightarrow \{0,1\}$ eine Indikatorfunktion, oder auch charakteristische Funktion mit den Werten $1$ bei allen
Punkte in $L$ und $0$ an den übrigen. Dadurch werden die durch den Datenterm der Energiefunktion festgelegten Bedingungen auf die Punkte beschränkt,
die auch tatsächlich zu einer Trajektorie gehören. Alle anderen Punkte können jeden beliebigen Wert annehmen.

Um die übrigen Punkte dazu zu bringen, nur spezielle Werte anzunehmen, wird eine Regularisierungsfunktion
\begin{equation}
  E_{\mathrm{reg}}(u) := \int_\Omega g \psi \left( \sum \limits_{i=1}^n \left| \nabla u_i \right|^2 \right) dx
\label{eq:reg1}
\end{equation}
eingeführt. Sie sorgt dafür, dass die Regionen einerseits kompakt und mit minimale Durchmesser bleiben,
aber auch, dass bevorzugt in Richtungen mit homogenen Bereichen propagiert wird. Ersters wird durch die regularisierte Norm der totalen Variation (TV Norm)
$\psi(s^2):=\sqrt{s^2+\varepsilon^2}$ mit $\varepsilon := 0.001$ erreicht. Anschaulich betrachtet, wird für ein minimales Auftreten von Kanten
$|\nabla u_i|$ in jedem der binären Labelbilder $u_i$ gesorgt und gleichzeitig eine Lösung ohne Kanten verhindert.
Zweiters wird durch die Diffusionsfunktion $g: \omega \rightarrow \mathbb{R}^+$
\begin{equation}
  g(|\nabla I(x)|^2) := \frac{1}{ \sqrt{|\nabla I(x)|^2+\varepsilon^2} }
\end{equation}
erreicht, die dafür sorgt, dass die Labelkanten vorzugsweise dort liegen, wo auch ein großer Farbgradient $|\nabla I(x)|$ liegt.

Die konvexe Kombination der Energien aus (\ref{eq:data1}) und (\ref{eq:reg1}) ergibt
\begin{equation}
  \begin{split}
    E(u) &:= \frac{\alpha}{2} \int_\Omega c \sum \limits_{i=1}^n (u_i - \tilde u_i)^2 dx \\
    & + (1-\alpha) \int_\Omega g \psi \left( \sum \limits_{i=1}^n \left| \nabla u_i \right|^2 \right) dx
  \end{split}
\label{eq:eng1}
\end{equation}
mit $\sum_i u_i(x) = 1,\; \forall x$, mit dem Steuerungsparameter $\alpha \in [0,1) $, der in Abhängigkeit zur Glaubwürdigkeit der
von den Trajektorien stammenden Labels gewählt werden kann. Für $\alpha \rightarrow 1$ wird die Minimierungsfunktion zur Interpolation, anderenfalls liegt eine
Approximation vor, die es erlaubt, fehlerhafte Labels zu korrigieren.

\subsection{Minimierung}

Um eine Variationsrechnung mit den Binärfunktionen $u_i$ durchführen zu können, muss zunächst das Problem relaxiert
betrachtet werden. Dazu wird vereinbart, dass $u_i$ jeden Wert im Interval $[0,1]$ annehmen kann. Diese Art von Relaxation wurde bereits in zahlreichen
ähnlichen Problemen vorgeschlagen \cite{008,018,014}. Die Euler-Lagrange Gleichungen, für die relaxierte Energiefunktion, sieht dann wie folgt aus:
\begin{equation}
  \begin{split}
    0 & = \alpha c (u_i - \tilde u_i) \\
    & - (1-\alpha) \mathrm{div}\left( g \psi' \left( \sum \limits_{i=1}^n \left| \nabla u_i \right|^2 \right) \nabla u_i \right) \quad \forall i
  \end{split}
\label{eq:lag1}
\end{equation}
Dieses nichtlineare System wird mittels Fixpunkt Iterationsschema gelöst, wobei der nichtlineare Faktor $\psi'(s^2) = (s^2 + \varepsilon^2)^{-\frac{1}{2}}$
in jeder Iteration konstant gehalten wird. Das daraus resultierende lineare System wird mit Hilfe des Überrelaxationsverfahrens (SOR) gelöst.
Die Bedingung $\sum_i u_i(x) = 1, \forall x$ wird in jedem Fixpunktiterationsschritt sicher gestellt, indem eine Normalisierung gemäß \cite{009}
durchgeführt wird. Das relaxierte Ergebnis wird schließlich nach $\{0,1\}^n$ projeziert, wobei
\begin{equation}
  u_i := \left\{
    \begin{split}
      1, & \quad \mathrm{if}\; i = \mathrm{argmax}_i\left\{u_i|=1,\dotsc,n\right\} \\
      0, & \quad \mathrm{else }
    \end{split} \right.
\end{equation}
gilt.


%%% Local Variables: 
%%% mode: latex
%%% TeX-master: "../main"
%%% End: 

\section{Exposition}

Lorem ipsum dolor sit amet, consetetur sadipscing elitr, sed diam
nonumy eirmod tempor invidunt ut labore et dolore magna aliquyam erat,
sed diam voluptua. At vero eos et accusam et justo duo dolores et ea
rebum. Stet clita kasd gubergren, no sea takimata sanctus est Lorem
ipsum dolor sit amet. Lorem ipsum dolor sit amet, consetetur
sadipscing elitr, sed diam nonumy eirmod tempor invidunt ut labore et
dolore magna aliquyam erat, sed diam voluptua. At vero eos et accusam
et justo duo dolores et ea rebum. Stet clita kasd gubergren, no sea
takimata sanctus est Lorem ipsum dolor sit amet. Lorem ipsum dolor sit
amet, consetetur sadipscing elitr, sed diam nonumy eirmod tempor
invidunt ut labore et dolore magna aliquyam erat, sed diam
voluptua. At vero eos et accusam et justo duo dolores et ea
rebum~\cite{ware:2004:IVP}. Stet clita kasd gubergren, no sea takimata
sanctus est Lorem ipsum dolor sit amet.

Duis autem vel eum iriure dolor in hendrerit in vulputate velit esse
molestie consequat, vel illum dolore eu feugiat nulla facilisis at
vero eros et accumsan et iusto odio dignissim qui blandit praesent
luptatum zzril delenit augue duis dolore te feugait nulla
facilisi. Lorem ipsum dolor sit amet, consectetuer adipiscing elit,
sed diam nonummy nibh euismod tincidunt ut laoreet dolore magna
aliquam erat volutpat~\cite{kindlmann:1999:SAG}.

Ut wisi enim ad minim veniam, quis nostrud exerci tation ullamcorper
suscipit lobortis nisl ut aliquip ex ea commodo
consequat~\cite{levoy:1989:DSV}. Duis autem vel eum iriure dolor in
hendrerit in vulputate velit esse molestie consequat, vel illum dolore
eu feugiat nulla facilisis at vero eros et accumsan et iusto odio
dignissim qui blandit praesent luptatum zzril delenit augue duis
dolore te feugait nulla facilisi.

Lorem ipsum dolor sit amet, consetetur sadipscing elitr, sed diam
nonumy eirmod tempor invidunt ut labore et dolore magna aliquyam erat,
sed diam voluptua. At vero eos et accusam et justo duo dolores et ea
rebum. Stet clita kasd gubergren, no sea takimata sanctus est Lorem
ipsum dolor sit amet. Lorem ipsum dolor sit amet, consetetur
sadipscing elitr, sed diam nonumy eirmod tempor invidunt ut labore et
dolore magna aliquyam erat, sed diam voluptua. At vero eos et accusam
et justo duo dolores et ea rebum. Stet clita kasd gubergren, no sea
takimata sanctus est Lorem ipsum dolor sit amet. Lorem ipsum dolor sit
amet, consetetur sadipscing elitr, sed diam nonumy eirmod tempor
invidunt ut labore et dolore magna aliquyam erat, sed diam
voluptua. At vero eos et accusam et justo duo dolores et ea
rebum. Stet clita kasd gubergren, no sea takimata sanctus est Lorem
ipsum dolor sit amet.

Lorem ipsum dolor sit amet, consetetur sadipscing elitr, sed diam
nonumy eirmod tempor invidunt ut labore et dolore magna aliquyam erat,
sed diam voluptua. At vero eos et accusam et justo duo dolores et ea
rebum. Stet clita kasd gubergren, no sea takimata sanctus est Lorem
ipsum dolor sit amet. Lorem ipsum dolor sit amet, consetetur
sadipscing elitr, sed diam nonumy eirmod tempor invidunt ut labore et
dolore magna aliquyam erat, sed diam voluptua. At vero eos et accusam
et justo duo dolores et ea rebum. Stet clita kasd gubergren, no sea
takimata sanctus est Lorem ipsum dolor sit amet. Lorem ipsum dolor sit
amet, consetetur sadipscing elitr, sed diam nonumy eirmod tempor
invidunt ut labore et dolore magna aliquyam erat, sed diam
voluptua. At vero eos et accusam et justo duo dolores et ea
rebum. Stet clita kasd gubergren, no sea takimata sanctus est Lorem
ipsum dolor sit amet.

Lorem ipsum dolor sit amet, consetetur sadipscing elitr, sed diam
nonumy eirmod tempor invidunt ut labore et dolore magna aliquyam erat,
sed diam voluptua. At vero eos et accusam et justo duo dolores et ea
rebum. Stet clita kasd gubergren, no sea takimata sanctus est Lorem
ipsum dolor sit amet. Lorem ipsum dolor sit amet, consetetur
sadipscing elitr, sed diam nonumy eirmod tempor invidunt ut labore et
dolore magna aliquyam erat, sed diam voluptua.


\begin{figure}[tb]
  \centering
  \includegraphics[width=1.5in]{sample}
  \caption{\label{fig:sample} Beispielillustration.}
\end{figure}

\begin{equation}
  \sum_{j=1}^{z} j = \frac{z(z+1)}{2}
\end{equation}

Lorem ipsum dolor sit amet, consetetur sadipscing elitr, sed diam
nonumy eirmod tempor invidunt ut labore et dolore magna aliquyam erat,
sed diam voluptua. At vero eos et accusam et justo duo dolores et ea
rebum. Stet clita kasd gubergren, no sea takimata sanctus est Lorem
ipsum dolor sit amet. Lorem ipsum dolor sit amet, consetetur
sadipscing elitr, sed diam nonumy eirmod tempor invidunt ut labore et
dolore magna aliquyam erat, sed diam voluptua. At vero eos et accusam
et justo duo dolores et ea rebum. Stet clita kasd gubergren, no sea
takimata sanctus est Lorem ipsum dolor sit amet. Lorem ipsum dolor sit
amet, consetetur sadipscing elitr, sed diam nonumy eirmod tempor
invidunt ut labore et dolore magna aliquyam erat, sed diam
voluptua. At vero eos et accusam et justo duo dolores et ea
rebum. Stet clita kasd gubergren, no sea takimata sanctus est Lorem
ipsum dolor sit amet.

Lorem ipsum dolor sit amet, consetetur sadipscing elitr, sed diam
nonumy eirmod tempor invidunt ut labore et dolore magna aliquyam erat,
sed diam voluptua. At vero eos et accusam et justo duo dolores et ea
rebum. Stet clita kasd gubergren, no sea takimata sanctus est Lorem
ipsum dolor sit amet. Lorem ipsum dolor sit amet, consetetur
sadipscing elitr, sed diam nonumy eirmod tempor invidunt ut labore et
dolore magna aliquyam erat, sed diam voluptua. At vero eos et accusam
et justo duo dolores et ea rebum. Stet clita kasd gubergren, no sea
takimata sanctus est Lorem ipsum dolor sit amet. Lorem ipsum dolor sit
amet, consetetur sadipscing elitr, sed diam nonumy eirmod tempor
invidunt ut labore et dolore magna aliquyam erat, sed diam
voluptua. At vero eos et accusam et justo duo dolores et ea
rebum. Stet clita kasd gubergren, no sea takimata sanctus est Lorem
ipsum dolor sit amet.

\begin{table}
  %% Table captions on top in journal version
  \caption{\label{tab:vis_accept} Vis Paper Acceptance Rate}
  \scriptsize
  \begin{center}
    \begin{tabular}{cccc}
      Year & Submitted & Accepted & Accepted (\%)\\
      \hline
      1994 &  91 & 41 & 45.1\\
      1995 & 102 & 41 & 40.2\\
      1996 & 101 & 43 & 42.6\\
      1997 & 117 & 44 & 37.6\\
      1998 & 118 & 50 & 42.4\\
      1999 & 129 & 47 & 36.4\\
      2000 & 151 & 52 & 34.4\\
      2001 & 152 & 51 & 33.6\\
      2002 & 172 & 58 & 33.7\\
      2003 & 192 & 63 & 32.8\\
      2004 & 167 & 46 & 27.6\\
      2005 & 268 & 88 & 32.8\\
      2006 & 228 & 63 & 27.6
    \end{tabular}
  \end{center}
\end{table}


Lorem ipsum dolor sit amet, consetetur sadipscing elitr, sed diam
nonumy eirmod tempor invidunt ut labore et dolore magna aliquyam erat,
sed diam voluptua. At vero eos et accusam et justo duo dolores et ea
rebum. Stet clita kasd gubergren, no sea takimata sanctus est Lorem
ipsum dolor sit amet. Lorem ipsum dolor sit amet, consetetur
sadipscing elitr, sed diam nonumy eirmod tempor invidunt ut labore et
dolore magna aliquyam erat, sed diam voluptua. At vero eos et accusam
et justo duo dolores et ea rebum. Stet clita kasd gubergren, no sea
takimata sanctus est Lorem ipsum dolor sit amet.

\begin{figure*}[tb]
  \centering
  \includegraphics[width=4cm]{images/foot1}\hfill
  \includegraphics[width=4cm]{images/foot2}\hfill
  \includegraphics[width=4cm]{images/foot3}\hfill
  \includegraphics[width=4cm]{images/foot4}
  \caption{Illustration "uber beide Textspalten hinweg. Auch
    Illustrationen m"ussen entsprechend den Quellen gekennzeichnet
    werden. \cite{strengert2006spectral}}
  \label{fig:multicolumn}
\end{figure*}


Lorem ipsum dolor sit amet, consetetur sadipscing elitr, sed diam
nonumy eirmod tempor invidunt ut labore et dolore magna aliquyam erat,
sed diam voluptua. At vero eos et accusam et justo duo dolores et ea
rebum. Stet clita kasd gubergren, no sea takimata sanctus est Lorem
ipsum dolor sit amet. Lorem ipsum dolor sit amet, consetetur
sadipscing elitr, sed diam nonumy eirmod tempor invidunt ut labore et
dolore magna aliquyam erat, sed diam voluptua. At vero eos et accusam
et justo duo dolores et ea rebum. Stet clita kasd gubergren, no sea
takimata sanctus est Lorem ipsum dolor sit amet. Lorem ipsum dolor sit
amet, consetetur sadipscing elitr, sed diam nonumy eirmod tempor
invidunt ut labore et dolore magna aliquyam erat, sed diam
voluptua. At vero eos et accusam et justo duo dolores et ea
rebum. Stet clita kasd gubergren, no sea takimata sanctus est Lorem
ipsum dolor sit amet.


\subsection{Mezcal Head}

Duis autem~\cite{Lorensen:1987:MCA} vel eum iriure dolor in hendrerit
in vulputate velit esse molestie consequat, vel illum dolore eu
feugiat nulla facilisis at vero eros et accumsan et iusto odio
dignissim qui blandit praesent luptatum zzril delenit augue duis
dolore te feugait nulla facilisi. Lorem ipsum dolor sit amet,
consectetuer adipiscing elit, sed diam nonummy nibh euismod tincidunt
ut laoreet dolore magna aliquam erat volutpat%
\footnote{Fu"snoten erscheinen an der Unterkante der Spalte. Sie
  sollten jedoch vermieden werden, da der Lesefluss gest"ort wird.}.


\subsubsection{Ejector Seat Reservation}

Ut wisi enim ad minim veniam, quis nostrud exerci tation ullamcorper
suscipit lobortis nisl ut aliquip ex ea commodo
consequat~\cite{Nielson:1991:TAD}. Duis autem vel eum iriure dolor in
hendrerit in vulputate velit esse molestie consequat, vel illum dolore
eu feugiat nulla facilisis at vero eros et accumsan et iusto odio
dignissim qui blandit praesent luptatum zzril delenit augue duis
dolore te feugait nulla facilisi.

\paragraph{Rejected Ejector Seat Reservation}

Ut wisi enim ad minim veniam, quis nostrud exerci tation ullamcorper
suscipit lobortis nisl ut aliquip ex ea commodo consequat. Duis autem
vel eum iriure dolor in hendrerit in vulputate velit esse molestie

\section{Zusammenfassung}

Die Arbeit von P. Ochs und T. Brox stellt ein variationales, hierarchisches Model vor, um aus dünnen, unvollständingen Sets von Labeln dichte Segmente
zu generieren. Der variationale Ansatz optimiert dabei gleichzeit kontinuerliche Funktionen auf mehreren Superpixelebenen, und ist, nach Aussagen
der Authoren, der erste variationale Ansatz, der auf mehreren Ebenen arbeitet. Bei verschiedenen Experimenten stellten die Authoren fest,
dass ihre Segmentierung sogar noch die Genauigkeit der unvollständingen Eingabesets verbessert. Ziel der Arbeit war es, die
Forschung in Richtung der unbeaufsichtigten Objektsegmentierung und somit letztendich auch dem vollständig unbeaufsichtigten Lernens voranzubringen.

%%% Local Variables: 
%%% mode: latex
%%% TeX-master: "../main"
%%% End: 


\bibliographystyle{abbrv}
%% use following if all content of bibtex file should be shown
% \nocite{*}
\bibliography{literatur}
\end{document}
