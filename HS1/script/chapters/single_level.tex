\section{Einstufiges, variationales Model}

Wie in den vorangegangen Sektionen beschreiben stehen nun die erwähnten Punkttrajektorien und Labelinformation zur Verfügung. Diese seien
beschrieben als Labelfunktion $\tilde u := (\tilde u_1,\dotsc,\tilde u_n): \Omega \rightarrow  \{ 0,1 \}^n, n \in \mathbb{N}$, die $n$ verschiedene
Labels repräsentiert, wobei
\begin{equation}
  \tilde u_i := \left\{
    \begin{split}
      1, & \quad \mathrm{if}\; x\in L_i \\
      0, & \quad \mathrm{else }
    \end{split} \right.
\end{equation}
und $L_i$ das Set von Koordinaten ist, die von einer Trajektorien des Labels $i$ eingenommen werden, und $\Omega \subset \mathbb{R}^2$ den
Bildbereich beschreibt. Anschaulich gesprochen würde $\tilde u$ eine Menge von Binärbildern beschreiben, bei dem jedes $\tilde u_i$ für ein Label steht.
Der Einfachheit halber wird sich im Folgenden auf die Arbeit mit Einzelbildern beschränkt. Jedoch versichern die
Authoren, dass die Methoden auch ohne weiteres auf die Berechnung der ganzen Videosequenz erweitert werden kann.

Gesucht ist nun eine Funktion $u := (u_1,\dotsc,u_n): \Omega \rightarrow  \{ 0,1 \}^n$, die nahe an den Labels bleibt, die bereits in den Punkten in
$L := \cup_{i=1}^n L_i$ verfügbar sind. Dies wird erreicht, indem man versucht die Energie
\begin{equation}
  E_{\mathrm{data}}(u) := \frac{1}{2} \int_\Omega c \sum \limits_{i=1}^n (u_i - \tilde u_i)^2 dx
\label{eq:data1}
\end{equation}
zu minimieren. Dabei ist $c:\Omega \rightarrow \{0,1\}$ eine Indikatorfunktion, oder auch charakteristische Funktion mit den Werten $1$ bei allen
Punkte in $L$ und $0$ an den übrigen. Dadurch werden die durch den Datenterm der Energiefunktion festgelegten Bedingungen auf die Punkte beschränkt,
die auch tatsächlich zu einer Trajektorie gehören. Alle anderen Punkte können jeden beliebigen Wert annehmen.

Um die übrigen Punkte dazu zu bringen, nur spezielle Werte anzunehmen, wird eine Regularisierungsfunktion
\begin{equation}
  E_{\mathrm{reg}}(u) := \int_\Omega g \psi \left( \sum \limits_{i=1}^n \left| \nabla u_i \right|^2 \right) dx
\label{eq:reg1}
\end{equation}
eingeführt. Sie sorgt dafür, dass die Regionen einerseits kompakt und mit minimale Durchmesser bleiben,
aber auch, dass bevorzugt in Richtungen mit homogenen Bereichen propagiert wird. Ersters wird durch die regularisierte Norm der totalen Variation (TV Norm)
$\psi(s^2):=\sqrt{s^2+\varepsilon^2}$ mit $\varepsilon := 0.001$ erreicht. Anschaulich betrachtet, wird für ein minimales Auftreten von Kanten
$|\nabla u_i|$ in jedem der binären Labelbilder $u_i$ gesorgt und gleichzeitig eine Lösung ohne Kanten verhindert.
Zweiters wird durch die Diffusionsfunktion $g: \omega \rightarrow \mathbb{R}^+$
\begin{equation}
  g(|\nabla I(x)|^2) := \frac{1}{ \sqrt{|\nabla I(x)|^2+\varepsilon^2} }
\end{equation}
erreicht, die dafür sorgt, dass die Labelkanten vorzugsweise dort liegen, wo auch ein großer Farbgradient $|\nabla I(x)|$ liegt.

Die konvexe Kombination der Energien aus (\ref{eq:data1}) und (\ref{eq:reg1}) ergibt
\begin{equation}
  \begin{split}
    E(u) &:= \frac{\alpha}{2} \int_\Omega c \sum \limits_{i=1}^n (u_i - \tilde u_i)^2 dx \\
    & + (1-\alpha) \int_\Omega g \psi \left( \sum \limits_{i=1}^n \left| \nabla u_i \right|^2 \right) dx
  \end{split}
\label{eq:eng1}
\end{equation}
mit $\sum_i u_i(x) = 1,\; \forall x$, mit dem Steuerungsparameter $\alpha \in [0,1) $, der in Abhängigkeit zur Glaubwürdigkeit der
von den Trajektorien stammenden Labels gewählt werden kann. Für $\alpha \rightarrow 1$ wird die Minimierungsfunktion zur Interpolation, anderenfalls liegt eine
Approximation vor, die es erlaubt, fehlerhafte Labels zu korrigieren.

\subsection{Minimierung}

Um eine Variationsrechnung mit den Binärfunktionen $u_i$ durchführen zu können, muss zunächst das Problem relaxiert
betrachtet werden. Dazu wird vereinbart, dass $u_i$ jeden Wert im Interval $[0,1]$ annehmen kann. Diese Art von Relaxation wurde bereits in zahlreichen
ähnlichen Problemen vorgeschlagen \cite{008,018,014}. Die Euler-Lagrange Gleichungen, für die relaxierte Energiefunktion, sieht dann wie folgt aus:
\begin{equation}
  \begin{split}
    0 & = \alpha c (u_i - \tilde u_i) \\
    & - (1-\alpha) \mathrm{div}\left( g \psi' \left( \sum \limits_{i=1}^n \left| \nabla u_i \right|^2 \right) \nabla u_i \right) \quad \forall i
  \end{split}
\label{eq:lag1}
\end{equation}
Dieses nichtlineare System wird mittels Fixpunkt Iterationsschema gelöst, wobei der nichtlineare Faktor $\psi'(s^2) = (s^2 + \varepsilon^2)^{-\frac{1}{2}}$
in jeder Iteration konstant gehalten wird. Das daraus resultierende lineare System wird mit Hilfe des Überrelaxationsverfahrens (SOR) gelöst.
Die Bedingung $\sum_i u_i(x) = 1, \forall x$ wird in jedem Fixpunktiterationsschritt sicher gestellt, indem eine Normalisierung gemäß \cite{009}
durchgeführt wird. Das relaxierte Ergebnis wird schließlich nach $\{0,1\}^n$ projeziert, wobei
\begin{equation}
  u_i := \left\{
    \begin{split}
      1, & \quad \mathrm{if}\; i = \mathrm{argmax}_i\left\{u_i|=1,\dotsc,n\right\} \\
      0, & \quad \mathrm{else }
    \end{split} \right.
\end{equation}
gilt.


%%% Local Variables: 
%%% mode: latex
%%% TeX-master: "../main"
%%% End: 
