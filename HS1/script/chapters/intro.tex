Aktuelle Lernansätze für visuelle Erkennung hängen sehr von der manuellen Markierung und Segmentierung von Objeckten ab.
Betrachtet man das bis heute beste visuelle Erkennungssystem - das menschliche Gehirn - wird klar,
dass solche manuellen Hilfestellungen nicht notwendig sind. Säuglinge erlernen die visuellen Formen und
Eigenschaften von Objekten auch ohne dass ihnen ihre Eltern Bounding Boxes darum legen oder eine Segmentierung zur Verfügung stellen.
Es gibt überzeugende Beweise darüber, dass Säuglinge dies Art von Objektsegmentierung durch Bewegungseinsatzen dürchführen \cite{}.
und man könnte letztendlich argumentieren, dass das rechnergestützte visuelle Systeme sich immer weiter dem menschlichen Sehvermögen annähren sollten.
\\
Die Bewegungsanalyse von Punkttrajektorien ist ein angemessens und robustes Werkzeug, um in Videosequenzen die Regionen von Objekten ohne
menschliches Zutun automatisch bestimmen und extrahieren zu können, wie in \cite{} erst kürzlich gezeigt.
Diese Ansätze verlangen jedoch danach, dass für die Bewegungsschätzung immer auch genügend Strukturen in den Bildern vorhanden sind,
zu denen Übereinstimmungen gefunden werden können.
In homogenen Gebieten gibt es diese Strukturen aber nicht, was dazu führt, dass die resultierenden Punkttrajektorien nur spärlich vorhanden sind.
In der Arbeit von \cite{} werden die Punkttrajektorien zwar aus dem dichten, optischen Flussfeld berechnet und resultierenden Trajektorien würden
ebenfalls für das gesamte Bild zur Verfügung stehen, jedoch sind diese in homogene Regionen weniger zuverlässig und können das Clustering behindern.
Zudem verlangen die eingeschränkte Verfügbarkeit von Rechenkraft nach der Reduzierung der Trajektorien, die analysiert werden sollen.
Das Clustern von dichten Punkttrajektorien würde viel zu lange dauern.
\\
Die Authoren stellten in ihrem Artikel eine Methode basierend auf der Variationsrechnung vor,
die aus wenigen Clustern von Punkttrajektorien eine dichte Segmentierung erstellt\ref{fig:01}.
Auf dem ersten Blick mag dies nach einem simplen Interpolationsproblem aussehen, da unser Verstand ganz einfach die Lücken zwischen den Punkten füllen kann.
Bei genaurer Betrachtung werden jedoch einige Schwierigkeiten deutlich.
So werden zum Beispiel einige der kritischen Bereich überhaupt nicht von den Trajektorien abgedeckt,
ganz besonders sind davon die Grenzen der Objekte betroffen. Den Trajektorien, die an den Grenzen jedoch vorhanden sind,
wurden in den meisten Fällen falsche Labels zugewiesen, da der zugrunde liegende optische Fluss gerade bei Okklusion ungenau wird.
Schlussendlich ist gerade in homogenen nahezu keine Information über die entsprechenden Labels vorhanden.
\\
Der Schlüssel, um die bestehenden Daten der Labels zuverlässing propagieren zu können, liegt darin, sich die Farb- und Kanteninformationen
zu nutze zu machen, was sich sehr gut mit der Trajektorienbestimmung ergänzt. Bemerkenswert ist, das Segmentierung basierend auf Farbdaten
am besten in homogenen Bereichen arbeitet, eben dort wo die Probleme der Bewegungsbasierten Segmentierung liegen.
Dies wird erreicht, indem die Informationen in Abhängigkeit der Farbhomogenität verbreitet wird. Am Ende wird ein hierarchischer variationaler Ansatz
vorgestellt, mit kontinuierliche Labelfunktion auf mehreren Ebenen. Jede Ebene entspricht einer Unterteilung in Superpixeln mit einem bestimmten
Grobkrörnigkeit. Im Vergleich zu einem Model mit nur einer Ebene stehen  Hilfsfunktionen auf den gröberen Ebenen zur Verfügung, die mittels eines
verbindenden Diffusionsprozesses optimiert werden.

Der Vorteil dieses Verfahrens liegt darin, dass das Propagieren der Labels, dank des hierarchischen Ansatzes, die Struktur berücksichtig, während Metrisierungsfehler und Blockartefakte vermieden werden können. Dieses treten besonders bei diskreten Markov Random Fields (MRF) Modellen auf.
