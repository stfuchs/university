Aktuelle Lernansätze für visuelle Erkennung hängen sehr von der manuellen Markierung und Segmentierung von Objeckten ab.
Betrachtet man das bis heute beste visuelle Erkennungssystem - das menschliche Gehirn - wird klar,
dass solche manuellen Hilfestellungen nicht notwendig sind. Säuglinge erlernen die visuellen Formen und
Eigenschaften von Objekten auch ohne dass ihnen ihre Eltern Bounding Boxes darum legen oder eine Segmentierung zur Verfügung stellen.
Es gibt überzeugende Beweise darüber, dass Säuglinge dies Art von Objektsegmentierung durch Bewegungseinsatzen dürchführen \cite{}.
und man könnte letztendlich argumentieren, dass das rechnergestützte visuelle Systeme sich immer weiter dem menschlichen Sehvermögen annähren sollten.

Die Bewegungsanalyse von Punkttrajektorien ist ein angemessens und robustes Werkzeug, um in Videosequenzen die Regionen von Objekten ohne
menschliches Zutun automatisch bestimmen und extrahieren zu können, wie in \cite{} erst kürzlich gezeigt. Diese Ansätze verlangen jedoch danach, dass für die Bewegungsschätzung immer auch genügend Strukturen in den Bildern vorhanden sind, zu denen Übereinstimmungen gefunden werden können.
In homogenen Gebieten gibt es diese Strukturen aber nicht, was dazu führt, dass die resultierenden Punkttrajektorien nur spärlich vorhanden sind.
In der Arbeit von \cite{} werden die Punkttrajektorien zwar aus dem dichten, optischen Flussfeld berechnet und resultierenden Trajektorien würden
ebenfalls für das gesamte Bild zur Verfügung stehen, jedoch sind diese in homogene Regionen weniger zuverlässig und können das Clustering behindern.