\section{Zusammenfassung}

Die Arbeit von P. Ochs und T. Brox stellt ein variationales, hierarchisches Model vor, um aus dünnen, unvollständingen Sets von Labeln dichte Segmente
zu generieren. Der variationale Ansatz optimiert dabei gleichzeit kontinuerliche Funktionen auf mehreren Superpixelebenen, und ist, nach Aussagen
der Authoren, der erste variationale Ansatz, der auf mehreren Ebenen arbeitet. Bei verschiedenen Experimenten stellten die Authoren fest,
dass ihre Segmentierung sogar noch die Genauigkeit der unvollständingen Eingabesets verbessert. Ziel der Arbeit war es, die
Forschung in Richtung der unbeaufsichtigten Objektsegmentierung und somit letztendich auch dem vollständig unbeaufsichtigten Lernens voranzubringen.

%%% Local Variables: 
%%% mode: latex
%%% TeX-master: "../main"
%%% End: 
