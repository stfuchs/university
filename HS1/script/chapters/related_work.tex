\section{Verwandte Arbeiten}


Die hier behandelte Problematik ist sehr verwandt mit der interaktiven Segmentierung,
bei der der Benutzer selbst zunächst einfache Markierungen im Bild vornimmt
und anschließend propagiert das entsprechend gewählte Verfahren diese Labels zu den restlichen Stellen.
Mehrere existierende Techniken basiern auf Graph Cuts \cite{004} oder Random Walks \cite{012}.
Die aktuellsten Techniken bauen auf konvex relaxierte Variationsrechnungen \cite{023,018,014,016} auf, die es, im Gegensatz zu den klassischen,
auf einem Graph definierten MRF, vermeiden, Diskretisierungsartefakte zu erzeugen.
Die hier vorgestellte variationale Methode baut auf die Regularizierung von \cite{014} auf.

Keines der genannten Verfahren zieht einen hierarchischen Ansatz in Betracht. Vielmehr unterscheiden sich die Labels
der Punkttrajektorien von denen, die durch die Benutzer eingezeichnet werden, in zwei Dingen:
Zum einen sind erstere unbeaufsichtig generiert und damit mit großer Wahrscheinlichkeit fehlerbehaftet, während bei letzteren stets
von der Richtigkeit der Benutzereingabe ausgegangen wird. Dies bedeutet, dass man nicht von einem Interpolationsproblem ausgeht,
sondern ein Approximationsproblem vorliegt.
Zum anderen erhält man durch den Benutzer eine dichte und endliche Vorauswahl, während die Labels der Trajektorien sich nur aus einzelnen Punkten zusammen
setzen und über das gesamte Bild verteilt sind.

Das hier vorgestellte Model ist ebenfalls verwandt mit dem Bildkompressionsverfahren durch anistrope Diffusion \cite{011}. Dieses
Verfahren erhält nur eine kleine Menge der Pixelwerte der ursprünglichen Bildpunkte und versucht die übrigen mittels
Diffusionsprozess wieder herzustellen.

Desweiteren gibt es zahlreiche, aktuelle Arbeiten zur dichten Bewegungssegmentierung, die eine Übersegmentierung durch den Einsatz von
Superpixeln, Labelpropagierung mittels optischen Fluss oder anderen Clusterverfahren erzielen \cite{005,013,024,015}.
Diese liefern jedoch nicht die tatsächlichen Objektregionen.
Einige interaktive Videosegmentierungen versuchen dies zu vermeiden \cite{003,019}, jedoch verlangen diese Verfahren wiederum die Eingabe eines
Benutzers und laufen somit nicht unbeaufsichtigt ab.