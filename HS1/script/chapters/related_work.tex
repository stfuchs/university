\section{Instruktionen}

Bitte nachfolgende Abschnitte sorgf"altig durchlesen.

\subsection{Sprache}

Die Ausarbeitung kann entweder auf Deutsch oder Englisch geschrieben
werden.

\subsection{Abbildungen und Tabellen}

Alle Abbildungen (siehe Abb.\ \ref{fig:sample}) und Tabellen (Tabelle\
\ref{tab:vis_accept}) sollten zentriert sein
(\verb|\centering|). Abbildungen "uber beide Textspalten
(Abb. \ref{fig:multicolumn}) k"onnen mit
\verb|\begin{figure*}|\ldots\verb|\end{figure*}| eingef"ugt werden.

\subsection{Referenzen}

Literaturangaben wie beispielsweise Levoy \cite{levoy:1989:DSV} werden
mit Hilfe von BibTeX erzeugt. Dazu werden die Referenzen in die
Literaturliste (hier \emph{literatur.bib}) eingetragen und
entsprechend mit \verb|\cite| referenziert.

\subsection{\LaTeX-"Ubersetzung}

Die \LaTeX-Datei kann mit \emph{latex} oder \emph{pdflatex} "ubersetzt
werden. Dabei ist zu beachten, dass f"ur die "Ubersetzung mit
\emph{latex} die Grafiken in Postscript (eps) vorliegen, f"ur
\emph{pdflatex} entsprechend als jpg, png oder pdf.  Der Ablauf ist
dabei der folgende:
\begin{enumerate}
\item \verb|pdflatex <quelldatei.tex>|
\item \verb|bibtex <quelldatei>|
\item \verb|pdflatex <quelldatei.tex>| (evtl. mehrfach)
\end{enumerate}
Alternativ kann auch das mitgelieferte Makefile verwendet werden.

